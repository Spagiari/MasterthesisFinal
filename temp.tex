
\section{PRÓXIMOS PASSOS}

Como sequência das atividades desse trabalho será desenvolvido um filtro para eliminar os falsos positivs em periodos de baixa atividade solar. Como parte produção principal desse trabalho as analises realizadas para o ano de 2003 mostrada na sessão de resultados seram extendidas para todo o periodo que compõem o ciclo solar 23. Essas analises terão como objetivo verificar as correlações entre area, intensidade, campo magnético das manchas encontradas por \citecustom{Kopp1992} e \citecustom{Penn2006} e também procurar por outras possíveis correlações. Os dados produzidos por também serão usados para verificar como as propriedades das manchas, tais como, area, intensidade e campo magnético variam ao longo do ciclo magnético.

Espera-se conseguir resultados de contagem de manchas solares e área de manchas solares com alto índice de correlação quando compradado com as fontes oficiais NOAA e SIDC. Espera-se também que os dados gerados tenham qualidade suficiente para verificar as correlações conhecidas entre as propriedades das manchas solares. Por final esperamos encontrar indícios de variações das propriedades das manchas solares ao longo do ciclo magnético que possam ajudar a revelar algum detalhe sobre o funcionamento do ciclo magnético ou sobre o comportamento do comportamento dos ciclos magnéticos solares.

